\documentclass[a4paper,10pt]{article}
\usepackage{CJK}
\usepackage{indentfirst}
\usepackage{graphicx}
\usepackage{bibentry,natbib}
\usepackage{fancyhdr}
\usepackage{lastpage}

\usepackage[top=2.54cm, bottom=2.54cm, left=3.18cm, right=3.18cm]{geometry}
\begin{document}
\begin{CJK}{UTF8}{song}
\begin{center}
\Large 线性代数(liner algebra)
\end{center}
\section{行列式}
行列式是研究线性代数的一个工具,它是为求解线性方程组而引入的,但在数学的其他分支应用也很方泛。
\subsection{二阶和三阶行列式}
\begin{enumerate}
\item 定义1: 符号
 $\left|\begin{array}{cc}a_{11}&a_{12} \\
                                      a_{21}&a_{22}\end{array} \right|$ 
称为二阶行列式,它由两行两列4个数组成,它代表一个算式,等于数 $a_{11}a_{22}-a_{12}a_{21}$ ;即:
\begin{quote}
$\left|\begin{array}{cc}
    a_{11}&a_{12} \\
    a_{21}&a_{22}
\end{array} \right| =a_{11}a_{22}-a_{12}a_{21} $
\end{quote}
其中$a_{ij}(i,j=1,2)$称为行列式的元素,第一个下标i表示第不行,第二个下标j表示第j列。$a_{ij}$就表示第i行第j列相交处的那个元素。
\item 定义2:符号D=
$\left|\begin{array}{ccc}
a_{11}&a_{12}&a_{13} \\
a_{21}&a_{22}&a_{23} \\
a_{31}&a_{32}&a_{33}  \\
\end{array}\right| $
称为三阶行列式,它由$3^{2}$个数组成,也代表一个算式,即:
\begin{quote}
$D=\left|\begin{array}{ccc}
a_{11}&a_{12}&a_{13} \\
a_{21}&a_{22}&a_{23} \\
a_{31}&a_{32}&a_{33}  \\
\end{array}\right|  \\
 \quad\quad =a_{11}a_{22}a_{33}+a_{12}a_{23}a_{31}+a_{13}a_{21}a_{32}-a_{13}a_{22}a_{31}-a_{11}a_{23}a_{32}-a_{12}a_{21}a_{33}
$
\end{quote}
\end{enumerate}
\subsection{n阶行列式}
n阶行列式由$n^{2}$个元素构成,记为:
\begin{quote}
$D=\left|\begin{array}{cccc}
a_{11}&a_{12}&\cdots{}&a_{1n} \\
ka_{21}&ka_{22}&\cdots{}&ka_{2n} \\
\vdots&\vdots&\ddots{}&\vdots \\
a_{n1}&a_{n2}&\cdots{}&a_{nn} \\          
\end{array}\right| $
\end{quote}

其中$a_{ij}(i,j=1,2,3,...,n)$称为行列式第i行第j列的元素 \\
特殊地:
\begin{quote}
$D=\left|\begin{array}{cccc}
a_{11}&0&\cdots{}&0\\
0&a_{22}&\cdots{}&0 \\
\vdots&\vdots&\ddots{}&\vdots \\
0&0&\cdots{}&a_{nn} \\          
\end{array}\right| $
\end{quote}
称为主对角行列式。
\begin{quote}
$D=\left|\begin{array}{cccc}
a_{11}&a_{12}&\cdots{}&a_{1n} \\
0&a_{22}&\cdots{}&a_{2n} \\
\vdots&\vdots&\ddots{}&\vdots \\
0&0&\cdots{}&a_{nn} \\          
\end{array}\right| $
\end{quote}
称为上三角行列式
\begin{quote}
$D=\left|\begin{array}{cccc}
a_{11}&0&\cdots{}&0 \\
a_{21}&a_{22}&\cdots{}&0 \\
\vdots&\vdots&\ddots{}&\vdots \\
a_{n1}&a_{n2}&\cdots{}&a_{nn} \\          
\end{array}\right| $
\end{quote}
称为下三角行列式,这三个行列式的值都为:$a_{11}a_{22}\cdots{}a_{nn}$。

\section{行列式的性质}
\subsection{转置行列式}
将行列式D的行与相应的列互换后得到的新的行列式,称为D的转置行列式,记为$D^{T}$。
即若:
\begin{quote}
$D=\left|\begin{array}{cccc}
a_{11}&a_{12}&\cdots{}&a_{1n} \\
a_{21}&a_{22}&\cdots{}&a_{2n} \\
\vdots&\vdots&\ddots{}&\vdots \\
a_{n1}&a_{n2}&\cdots{}&a_{nn} \\          
\end{array}\right| $
\end{quote}
则转置行列式为:
\begin{quote}
$D^{T}=\left|\begin{array}{cccc}
a_{11}&a_{21}&\cdots{}&a_{n1} \\
a_{12}&a_{22}&\cdots{}&a_{n2} \\
\vdots&\vdots&\ddots{}&\vdots \\
a_{1n}&a_{2n}&\cdots{}&a_{nn} \\          
\end{array}\right| $
\end{quote}

\subsection{行列式的性质}
行列式具有如下性质:
\begin{description}
\item[性质1:] 行列式转置后,其值不变,即$D=D^{T}$;
\item[性质2:] 互换行列式中的任意两行(列),行列式仅改变符号;
\item[性质3:] 如果行列式中有两行(列)的对应元素相同,则此行列式为零;
\item[性质4:] 如果行列式中有一行(列)元素全为零,则这个行列式等于零;
\item[性质5:] 把行列式的某一行(列)的每一个元素同乘以数k,等于以数k乘该行列式,即:
\begin{quote}
$\left|\begin{array}{cccc}
a_{11}&a_{12}&\cdots{}&a_{1n} \\
ka_{21}&ka_{22}&\cdots{}&ka_{2n} \\
\vdots&\vdots&\ddots{}&\vdots \\
a_{n1}&a_{n2}&\cdots{}&a_{nn} \\          
\end{array}\right| 
=k
\left|\begin{array}{cccc}
a_{11}&a_{12}&\cdots{}&a_{1n} \\
a_{21}&a_{22}&\cdots{}&a_{2n} \\
\vdots&\vdots&\ddots{}&\vdots \\
a_{n1}&a_{nn}&\cdots{}&a_{nn} \\          
\end{array}\right|$
\end{quote}
\begin{description}
\item[推论1:] 如果行列式(列)的所有元素有公因子,则公因子可以提到行列式外面;
\item[推论2:] 如果行列式有两行(列)的对应成比列,则行列式等于零;
\end{description}
\item[性质6:] 如果行列式中的某一行(列)所有元素都是两个数的和,则此行列式等于两个行列式的和,而且这两个行列式除了这一行(列)以外,其余的元素与原行列式的对应元素相同,即:
\begin{quote}
$D=\left|\begin{array}{cccc}
a_{11}&a_{12}&\cdots{}&a_{1n} \\
a_{21}+b_{21}&a_{22}+b_{22}&\cdots{}&a_{2n}+b_{2n} \\
\vdots&\vdots&\ddots{}&\vdots \\
a_{n1}&a_{n2}&\cdots{}&a_{nn} \\          
\end{array}\right|\\
=
\left|\begin{array}{cccc}
a_{11}&a_{12}&\cdots{}&a_{1n} \\
a_{21}&a_{22}&\cdots{}&a_{2n} \\
\vdots&\vdots&\ddots{}&\vdots \\
a_{n1}&a_{n2}&\cdots{}&a_{nn} \\          
\end{array}\right|
+ 
\left|\begin{array}{cccc}
a_{11}&a_{12}&\cdots{}&a_{1n} \\
b_{21}&b_{22}&\cdots{}&b_{2n} \\
\vdots&\vdots&\ddots{}&\vdots \\
a_{n1}&a_{n2}&\cdots{}&a_{nn} \\          
\end{array}\right| $
\end{quote}
\item[性质7:] 以数k乘行列式的某一行(列)的所有元素,然后加到另一行(列)的对应元素上,则行列式的值不变,即:
\begin{quote}
$ 
\left|\begin{array}{cccc}
a_{11}&a_{12}&\cdots{}&a_{1n} \\
a_{21}&a_{22}&\cdots{}&a_{2n} \\
\vdots&\vdots&\ddots{}&\vdots \\
a_{n1}&a_{n2}&\cdots{}&a_{nn} \\          
\end{array}\right| 
=
\left|\begin{array}{cccc}
a_{11}&a_{12}&\cdots{}&a_{1n} \\
ka_{11}+a_{21}&ka_{12}+a_{22}&\cdots{}&ka_{1n}+a_{2n} \\
\vdots&\vdots&\ddots{}&\vdots \\
a_{n1}&a_{n2}&\cdots{}&a_{nn} \\          
\end{array}\right| 
$
\end{quote}

\end{description} 
\subsection{行列式的展开}
为了介绍行列式的展开,先引入余子式和代数余子式的概念。
\begin{description}
\item[定义:] 在n阶行列式中划去元素$a_{ij}$所在的第i行和第j列的元素,剩下的元素按原次序构成的 n-1 阶行列式称为$a_{ij}$的余子式,记和$M_{ij}$。
$a_{ij}$的余子式乘上$(-1)^{i+j}$称为$a_{ij}$的代数余子式,记作$A_{ij}$,即$A_{ij}=(-1)^{i+j}M_{ij}$。
\item[定理:] 行列式D等于它的任一行的各元素与对应的代数余子式的乘积之后,即:
\begin{displaymath}
D=a_{i1}A_{i1}+a_{i2}A_{i2}+\cdots{}+a_{in}A_{in}=\sum_{j=1}^{n}a_{ij}A_{ij} 
\end{displaymath}
或
\begin{displaymath}
D=a_{1j}A_{1j}+a_{2j}A_{2j}+\cdots{}+a_{nj}A_{nj}=\sum_{i=1}^{n}a_{ij}A_{ij}
\end{displaymath}
\item[推论:] 行列式的某一行(列)的元素与另一行(列)对应元素的代数余子式乘积之和等于零,即:
\begin{quote}
\begin{math}
a_{i1}A_{j1}+a_{i2}A_{j2}+\cdots{}+a_{in}A_{jn}=0, i\neq{}j \\
a_{1i}A_{1j}+a_{2i}A_{2j}+\cdots{}+a_{ni}A_{nj}=0, i\neq{}j
\end{math}
\end{quote}
结合定理可以写为:
\begin{displaymath}
\sum_{k=1}^{n}a_{ik}A_{jk}=\sum_{k=1}^{n}a_{ki}A_{kj}=
\biggl\{ \begin{array}{c}
D,i=j \\
0,i\neq{}j 
\end{array} 
\end{displaymath}
\end{description}
\section{克萊姆法则}
含有n个未知量的n个方程的线性方程组为:
\begin{quote}
\begin{math}
\left\{
\begin{array}{c}
a_{11}x_{1}+a_{12}x_{2}+\cdots{}+a_{1n}x_{n}=b_{1} \\
a_{21}x_{1}+a_{22}x_{2}+\cdots{}+a_{1n}x_{n}=b_{2} \\
\quad\quad \cdots{} \\
a_{n1}x_{1}+a_{n2}x_{2}+\cdots{}+a_{nn}x_{n}=b_{n} \\
\end{array} \right.
\end{math}
\end{quote}
将线性方程组系数组成的行列式记为D,即:
\begin{quote}
$D=
\left|\begin{array}{cccc}
a_{11}&a_{12}&\cdots{}&a_{1n} \\
a_{21}&a_{22}&\cdots{}&a_{2n} \\
\vdots&\vdots&\ddots{}&\vdots \\
a_{n1}&a_{n2}&\cdots{}&a_{nn} \\          
\end{array}\right| 
$
\end{quote}
用常数项$b_{1},b_{2},\cdots{},b_{n}$代替D中的第j列,组成的行列式记为$D_{j}$,即:
\begin{quote}
\begin{math}
D_{j}=\left|
\begin{array}{ccccccc}
a_{11}& \cdots{} & a_{1j-1} & b_{1} & a_{1j+1} & \cdots{} & a_{1n} \\
a_{21}& \cdots{} & a_{2j-1} & b_{2} & a_{2j+1} & \cdots{} & a_{2n} \\
\vdots{}&        & \vdots{} & \vdots{} & \vdots{} &       & \vdots{} \\
a_{n1}& \cdots{} & a_{nj-1} & b_{n} & a_{nj+1} & \cdots{} & a_{nn} \\
\end{array}\right| (j=1,2,\cdots{},n)
\end{math}
\end{quote}
\begin{description}
 \item[定理:](克莱姆法则)若线性方程组(1)的系数行列式$D\neq{}0$,则存在唯一解:
\begin{displaymath}
 x_{1}=\frac{D_{1}}{D},x_{2}=\frac{D_{2}}{D},\cdots{},x_{n}=\frac{D_{n}}{D}
\end{displaymath}
即:
\begin{displaymath}
x_{j}=\frac{D_{j}}{D} (j=1,2,\cdots{},n)
\end{displaymath}
\end{description}
克莱姆法则揭示了线性方程组的解与它的系数和常数项之间关系,用克莱姆法则解n元线性方程组有两个前提条件:
\begin{enumerate}
 \item 方程个数与未知数个数相等
\item 系数行列式D不等于零
\end{enumerate}

\section{矩阵的概念}
由$m\times{}n$个数$a_{ij}(i=1,2,\cdots,m,j=1,2,\cdots,n)$排成的矩形数表
\begin{quote}
$
\left[\begin{array}{cccc}
a_{11}&a_{12}&\cdots{}&a_{1n} \\
a_{21}&a_{22}&\cdots{}&a_{2n} \\
\vdots&\vdots&\ddots{}&\vdots \\
a_{m1}&a_{m2}&\cdots{}&a_{mn} \\    
\end{array}\right]     
$
\end{quote}
叫做一个m行n列的矩阵,简称$m\times{}n$矩阵,这个$m\times{}n$叫做矩阵的元素,$a_{ij}$称为该矩阵第i行第j列的元素。$m\times{}n$矩阵可记作$A_{m\times{}n}$或$(a_{ij}m\times{}n$,有时简记作A或$(a_{ij})$。
\begin{enumerate}
\item 当m=n时,矩阵A称为n阶方阵,矩阵与行列式是两个不同的概念,二都有本质区别。行列式可以展开,它的值是一个算式或一个数,矩阵是一个数表,它不表示一个算式或一个数与没有展开式,通常把方阵$A_{n\times{}n}$的元素按原来顺序所构成的行列式,称为方阵$A_{n\times{}n}$行列式,记作detA。如果方阵A满足$detA\neq{}0$,称A为非奇异方阵,否则,称A为奇异方阵。
\item 当m=1时,矩阵A称为行距阵;
\item 当n=1时,矩阵A称为列矩阵;
\item 如果矩阵的元素全为零,称A为零矩阵,记作0;
\item 在n阶方阵中,如果主对角线工下方的元素全为零,则称为上三角矩阵;
\item 在n阶方阵中,如果主对角线右下方的元素全为零,则称为下三角矩阵;
\item 如果一个方阵主对角线以外的元素全为零,则这个方阵称为对角方阵;
\item 在n阶对角方阵中,当对角线上的元素都为1时,则称为n称单位矩阵,记作E;
\item 如果矩阵$A=(a_{ij},B=(b_{ij})$的行数与列数分别相同,并且各对应位置的元素也相等,则称矩阵A与矩阵B相等,记作A=B;
\end{enumerate}
\subsection{转置矩阵}
将$m\times{}n$矩阵$A_{m\times{}n}$的行换成列,列换成行,所得到的$n\times{}m$矩阵称为$A_{m\times{}n}$的转置矩阵,记作$A^{T}$,即: \\
若
\begin{quote}
$A=\left[\begin{array}{cccc}
a_{11}&a_{12}&\cdots{}&a_{1n} \\
a_{21}&a_{22}&\cdots{}&a_{2n} \\
\vdots&\vdots&\ddots{}&\vdots \\
a_{m1}&a_{m2}&\cdots{}&a_{mn} \\          
\end{array}\right] $
\end{quote}
则$A^{T}$为
\begin{quote}
$A^{T}=\left[\begin{array}{cccc}
a_{11}&a_{21}&\cdots{}&a_{m1} \\
a_{12}&a_{22}&\cdots{}&a_{m2} \\
\vdots&\vdots&\ddots{}&\vdots \\
a_{1n}&a_{2n}&\cdots{}&a_{mn} \\          
\end{array}\right] $
\end{quote}
转置矩阵具有下列性质:
\begin{enumerate}
\item $(A^{T})^{T}=A$
\item $(A+B)^{T}=A^{T}+B^{T}$
\item $(\lambda{}A)^{T}=\lambda{}A^{T}$
\item $(AB)^T=B^{T}A^{T}$
\end{enumerate}


\section{矩阵的运算}
\subsection{矩阵的加法运算}
\begin{description}
 \item[定义:] 设两个$m\times{}n$矩阵$A=(a_{ij}),B=(b_{ij})$,将其对应位置元素相加(或相减)得到的$m\times{}n$,称为矩阵与矩阵的和(或差),记作$A\pm{}B$,即:
\begin{quote}
$A=\left[\begin{array}{cccc}
a_{11}&a_{12}&\cdots{}&a_{1n} \\
a_{21}&a_{22}&\cdots{}&a_{2n} \\
\vdots&\vdots&\ddots{}&\vdots \\
a_{m1}&a_{m2}&\cdots{}&a_{mn} \\          
\end{array}\right]  \quad 
B=\left[\begin{array}{cccc}
b_{11}&b_{12}&\cdots{}&b_{1n} \\
b_{21}&b_{22}&\cdots{}&b_{2n} \\
\vdots&\vdots&\ddots{}&\vdots \\
b_{m1}&b_{m2}&\cdots{}&b_{mn} \\          
\end{array}\right]  \\
A+B=\left[\begin{array}{cccc}
a_{11}+b_{11}&a_{12}+b_{12}&\cdots{}&a_{1n}+b_{1n} \\
a_{21}+b_{21}&a_{22}+b_{22}&\cdots{}&a_{2n}+b_{2n} \\
\vdots&\vdots&\ddots{}&\vdots \\
a_{m1}+b_{m1}&a_{m2}+b_{m2}&\cdots{}&a_{mn}+b_{mn} \\          
\end{array}\right] 
$
\end{quote}
注意:只有在两个矩阵的行数和列数都对应相同时才能作加法(或减法)运算。 \\
由定义,可得矩阵的加法具有以下性质:
\begin{enumerate}
\item $A+B=B+A$
\item $(A+B)+C=A+(B+C)$
\item $A+0=A$
\end{enumerate}
其中A、B、C、0都是$m\times{}n$矩阵.
\end{description}

\subsection{数与矩阵的乘法}
\begin{description}
\item[定义:]设k为任意数,以数k乘以矩阵A中每一个元素所得到的矩阵叫做k与A的积,记为kA (或Ak) 即:
\begin{quote}
$kA=(ka_{ij})_{m\times{}n}=
\left[\begin{array}{cccc}
ka_{11}&ka_{12}&\cdots{}&ka_{1n} \\
ka_{21}&ka_{22}&\cdots{}&ka_{2n} \\
\vdots&\vdots&\ddots{}&\vdots \\
ka_{m1}&ka_{m2}&\cdots{}&ka_{mn} \\          
\end{array}\right]  
$ 
\end{quote}
数乘具有以下运算规律:
\begin{enumerate}
\item $k(A+B)=kA+kB$
\item $(k+h)A=kA+hA$
\item $(kh)A=k(hA)$
\end{enumerate}
\end{description}
\subsection{矩阵与矩阵相乘}
\begin{description}
 \item[定义:]设矩阵$A=(a_{ik})_{m\times{}s},B=(b_{kj})_{s\times{}n}$(A的列数与B的行数相等),那么矩阵$C=(c_{ij})_{m\times{}n}$ 称为矩阵A与矩阵B的乘积,记做$C=A+B$其中:
\begin{displaymath}
c_{ij}=a_{i1}b_{1j}+a_{i2}b_{2j}+\cdots{}+a_{is}b_{sj}
=\sum_{k=1}^{s}a_{ik}b_{kj}(i=1,2,\cdots{},m,j=1,2,\cdots{},n)
\end{displaymath}
矩阵的乘法满足下列运算规律:
\begin{enumerate}
\item $(AB)C=A(BC)$
\item $A(B+C)=AB+AC$ , $(B+C)A=BA+CA$
\item $k(AB)=(kA)B=A(kB)$
\item $AE=EA=A$
\item $A^{k}=\underbrace{AA\cdots{}A}_{k}$ , $A^{k}A^{l}=A^{k+l}$ , $(A^{k})^{l}=A^{kl}$
\end{enumerate}
\end{description}
\subsection{矩阵的初等变换}
\begin{description}
\item[定义:]对矩行的行(或列)作下列三种变换称为矩阵的初等变换:
\begin{enumerate}
\item 位置变换:交换矩阵的某两行或列,用记号$r_{i}\leftrightarrow{}r_{j}$或$c_{i}\leftrightarrow{}c_{j}$表示;
\item 倍法变换:用一个不为零的数乘矩阵的某一行或列,用记号$kr_{i}$或$kc_{i}$表示;
\item 倍加变换:用一个数乘矩阵的某一行或列加到另一行或列上,用记号$kr_{i}+r_{j}$或$kc_{i}+c_{j}$表示;
\end{enumerate}

\end{description}

\section{逆矩阵}
\begin{description}
 \item[定义:]对于n阶方阵A,如果存在n阶方阵C,使得$AC=CA=E$(E为n阶单位矩阵),则把方阵C称为A的逆矩阵(简称逆阵)记作$A^{-1}$,即$C=A^{-1}$。\\
逆矩阵具有如下性质:
\begin{enumerate}
\item 若A是可逆的,则逆矩阵唯一;
\item 若A可逆,则$(A^{-1})^{-1}=A$;
\item 若A、B为同阶方阵且均可逆,则AB可逆,且$(AB)^{-1}=B^{-1}A^{-1}$;
\item 若A可逆,则$detA\neq{}0$。反之,若$detA\neq{}0$,则A是可逆的。
\end{enumerate}
\end{description}
\subsection{逆矩阵的求法}
\subsubsection{伴随矩阵}
设矩阵
\begin{quote}
$A=\left[\begin{array}{cccc}
a_{11}&a_{12}&\cdots{}&a_{1n} \\
a_{21}&a_{22}&\cdots{}&a_{2n} \\
\vdots&\vdots&\ddots{}&\vdots \\
a_{n1}&a_{n2}&\cdots{}&a_{nn} \\          
\end{array}\right]   $
\end{quote}
所对应的行列式detA中元素$a_{ij}$的代数余子式矩阵:
\begin{quote}
$\left[\begin{array}{cccc}
A_{11}&A_{21}&\cdots{}&A_{n1} \\
A_{12}&A_{22}&\cdots{}&A_{nn} \\
\vdots&\vdots&\ddots{}&\vdots \\
A_{1n}&A_{2n}&\cdots{}&A_{nn} \\          
\end{array}\right]   $
\end{quote}
称为A的伴随矩阵,记为$A^{*}$。 \\
显然:
\begin{quote}
$AA^{*}=\left[\begin{array}{cccc}
a_{11}&a_{12}&\cdots{}&a_{1n} \\
a_{21}&a_{22}&\cdots{}&a_{2n} \\
\vdots&\vdots&\ddots{}&\vdots \\
a_{n1}&a_{n2}&\cdots{}&a_{nn} \\          
\end{array}\right]
\left[\begin{array}{cccc}
A_{11}&A_{21}&\cdots{}&A_{n1} \\
A_{12}&A_{22}&\cdots{}&A_{nn} \\
\vdots&\vdots&\ddots{}&\vdots \\
A_{1n}&A_{2n}&\cdots{}&A_{nn} \\          
\end{array}\right] $
\end{quote}
仍是一个n阶方阵,其中第i行第j列的元素为:
\begin{quote}
$a_{i1}A_{j1}+a_{i2}A_{j2}+\cdots{}+a_{in}A_{jn}$
\end{quote}
由行列式按一行或列展开可知:
\begin{quote}
\begin{math}
\sum_{k=1}^{n}a_{ik}A_{jk}=\sum_{k=1}^{n}a_{ki}A_{kj}=
\biggl\{ \begin{array}{c}
D,i=j \\
0,i\neq{}j 
\end{array} 
\end{math}
\end{quote}
所以
\begin{quote}
$AA^{*}=\left[
\begin{array}{cccc}
detA& 0 & \cdots{} & 0 \\
0 & detA & \cdots{} & 0 \\
\vdots{} & \vdots{} & \ddots{} & \vdots{} \\
0 & 0 & \cdots{} & detA \\
\end{array}\right]
=detAE
$
\end{quote}
同理:
\begin{quote}
$AA^{*}=detAE=A^{*}A$   (1)
\end{quote}
\begin{description}
\item[定理:]n阶方阵A可逆的充分必要条件是A为非奇异矩阵,而且:
\begin{quote}
$
A^{-1}=\frac{1}{detA}A^{*}=\frac{1}{detA} 
\left[\begin{array}{cccc}
A_{11}&A_{21}&\cdots{}&A_{n1} \\
A_{12}&A_{22}&\cdots{}&A_{nn} \\
\vdots&\vdots&\ddots{}&\vdots \\
A_{1n}&A_{2n}&\cdots{}&A_{nn} \\          
\end{array}\right] $
\end{quote}
\item[必要性:]
如果A可逆,且 $A^{-1}$ 存在使 $AA^{-1}=E$ ,两边取行列式 $det(AA^{-1})=detE$ ,即 $detAdetA^{-1}=1$ ,因而 $detA\neq{}0$,即A为非奇异矩阵。
\item[充分必:]
设A为非奇异矩阵,所以$detA\neq{}0$,由(1)式可知$A(\frac{1}{detA}A^{*})=(\frac{1}{detA}A^{*})A=E$,所以A是可逆矩阵,且:
\begin{displaymath}
A^{-1}=\frac{1}{detA}A^{*}
\end{displaymath}

\end{description}





































\end{CJK}
\end{document}
