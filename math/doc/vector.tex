\documentclass[a4paper,10pt]{article}
\usepackage{CJK}
\usepackage{indentfirst}
\usepackage{graphicx}
\usepackage{bibentry,natbib}
\usepackage{fancyhdr}
\usepackage{lastpage}

\usepackage[top=2.54cm, bottom=2.54cm, left=3.18cm, right=3.18cm]{geometry}
\begin{document}
\begin{CJK}{UTF8}{song}
\begin{center}
\Large 向量(vector)与矩阵(matrix)
\end{center}
\section{向量}
\subsection{向量的定义}
在数学中,向量的定义如下:把即有大小又有方向的量叫作向量。由n个数组成的有序数组($a_{1},a_{2},....a_{n}$ )叫做一个n维向量,其中 $a_i$ (i=1,2,3,..,n) 叫作它的第i个分量。向量有两种表示方法,横向和纵向,通常把:
\begin{quote}
$a=(a_1,a_2,...,a_n)$叫做行向量,把$\left( \begin{array}{c} a_1 \\ a_2 \\ \vdots{} \\ a_n \end{array} \right)$叫做列向量
\end{quote}
\subsection{向量的运算}
设向量$\alpha=(\alpha{}_{1},\alpha{}_{2},\ldots{},\alpha{}_{n})$ , $\beta=(\beta{}_{1},\beta{}_{2},\ldots{},\beta{}_{n})$。
\begin{enumerate}
 \item 相等:当且仅当 $\alpha{}_{i}=\beta{}_{i} (i=1,2,\ldots{},n)$ 时,$\alpha=\beta$。
\item 向量$\alpha$与向量$\beta$相加、减:
\begin{displaymath}
\alpha{}\pm{}\beta{}=(\alpha{}_{1}\pm{}\beta{}_{1},\alpha{}_{2}\pm{}\beta{}_{2},\ldots{},\alpha{}_{n}\pm{}\beta{}_{n})
\end{displaymath}
\item 与标量k与向量$\alpha$相乘:
\begin{displaymath}
k\alpha=(k\alpha{}_{1},k\alpha{}_{2},\ldots{},k\alpha{}_{n})
\end{displaymath}
\item 计算向量$\alpha$的长度:
\begin{displaymath}
|\alpha|=\sqrt{\alpha_{1}^{2}+\alpha_{2}^{2}+\ldots{}+\alpha_{n}^{2}}
\end{displaymath}
\item 归一化(normalize):
\begin{displaymath}
\alpha'=\frac{\alpha}{|\alpha|}
\end{displaymath}
\item 向量$\alpha$与向量$\beta$的点积:
\begin{displaymath}
\alpha\cdot\beta=\sum_{i=1}^{n}\alpha_{i}\cdot\beta{i}=\alpha_{1}\cdot\beta_{1}+\alpha_{2}\cdot\beta_{2}+\ldots{}+\alpha_{n}\cdot\beta_{n}
\end{displaymath}
在几何空间中,点积可以直观的定义为(其中$\theta$表示两个向个向量这间的角度):
\begin{displaymath}
 \alpha\cdot\beta=|\alpha||\beta|cos\theta
\end{displaymath}
如果给定两个向量,它们之间的夹角可以通过下列公式得到:
\begin{displaymath}
cos\theta=\frac{\alpha\cdot\beta}{|\alpha||\beta|}
\end{displaymath}
\item 向量$\alpha$在向量$\beta$上的投影:

\begin{displaymath}
Proj_{\beta}\;\alpha=\frac{(\alpha\cdot\beta)*\beta}{|\beta||\beta|}
\end{displaymath}
\item 点积的乘法定律,如果给定向量u、v和w以及标量k;
\begin{enumerate}
\item $v\cdot u=u\cdot v $
\item $u \cdot (v+w)=u \cdot v + u \cdot w $
\item $k*(u\cdot v)=(k*u)\cdot v=u\cdot(k*v) $
\end{enumerate}
\item 设三维向量$u=(u_{x},u_{y},u_{z})$与$v=(v_{x},v_{y},v_{z})$,它们的叉积定义如下:
\begin{displaymath}
u\times v=|u||v|cos\theta\: n
\end{displaymath}
其中$\theta$为向量u与向量v的角度$(0^\circ\leq\theta\leq 180^\circ)$,而n是一个与u、v所在平面均垂直的单位矢量。向量u与向量v的另一种算法为,建立这样一个矩阵: 
\begin{quote}
\begin{math}
\left| \begin{array}{ccc}
	i & j & k \\
	u_{x} & u_{y} & u_{z} \\
	v_{x} & v_{y} & v_{z} \\
       \end{array}
\right|  
\quad\Rightarrow\quad
\alpha \times \beta =(u_{y}\cdot v_{z}-v_{y}\cdot u_{z} 
                     , v_{x}\cdot u_{z}-u_{x}\cdot v_{z} 
                     , u_{x}\cdot v_{y}-v_{x}\cdot u_{y})
\end{math}
\end{quote}
其中i,j,k分别是与x轴,y轴,z轴平行的单位矢量。

叉积的乘法定律,给定向量u、v和w以及标量k:
\begin{enumerate}
\item $u\times v= -(v \times u ) $
\item $u \times ( v +w)=(u\times v) + (u\times w) $
\item $(u+v)\times w =(u \times w) + (v \times w) $
\item $k*(u \times v) = (k*u)\times v =u \times(k*v)$
\end{enumerate}
\end{enumerate}
\section{矩阵}
\subsection{矩阵的定义}
矩阵是一个矩形阵列,有指定的行和列,通常说矩阵为$m\times n$,表示它有m行和n列,表示为:
\begin{quote}
\begin{math}
A=\left|
\begin{array}{cccc}
a_{11} & a_{12} & \cdots & a_{1m} \\
a_{21} & a_{22} & \cdots & a_{2m} \\
\vdots & \vdots & \ddots & \vdots \\
a_{n1} & a_{n2} & \cdots & a_{nm} \\
\end{array}
   \right|
\end{math}
\end{quote}
在3D图形学中,常用的矩阵为$2\times 2$与$3\times 3$矩阵。
\subsection{矩阵的运算}
设$m\times n$矩阵$
A=\left|
\begin{array}{cccc}
a_{11} & a_{12} & \cdots & a_{1m} \\
a_{21} & a_{22} & \cdots & a_{2m} \\
\vdots & \vdots & \ddots & \vdots \\
a_{n1} & a_{n2} & \cdots & a_{nm} \\
\end{array}
   \right|
$
与矩阵$
B=\left|
\begin{array}{cccc}
b_{11} & b_{12} & \cdots & b_{1m} \\
b_{21} & b_{22} & \cdots & b_{2m} \\
\vdots & \vdots & \ddots & \vdots \\
b_{n1} & b_{n2} & \cdots & b_{nm} \\
\end{array}
   \right|
$

\begin{enumerate}
 \item 矩阵A与B加法与减法:
\begin{quote}
\begin{math}
A\pm B=\left|
\begin{array}{cccc}
a_{11} \pm b_{11}& a_{12}\pm b_{11}& \cdots & a_{1m} \pm b_{1m} \\
a_{21} \pm b_{21}& a_{22}\pm b_{22}& \cdots & a_{2m} \pm b_{2m} \\
\vdots & \vdots & \ddots & \vdots \\
a_{n1}\pm b_{n1} & a_{n2}\pm b_{n2} & \cdots & a_{nm}\pm b_{nm}\\
\end{array}
\right|
\end{math}
\end{quote}
\item 矩阵的转置,将矩阵的行转换为列,例如将矩阵A转置后,记为$A^{T}$:
\begin{quote}
\begin{math}
A^{T}=
\left|
\begin{array}{cccc}
a_{11} & a_{21} & \cdots & a_{n1} \\
a_{12} & a_{22} & \cdots & a_{n2} \\
\vdots & \vdots & \ddots & \vdots \\
a_{1m} & a_{2m} & \cdots & a_{nm} \\
\end{array}
\right|
\end{math}
\end{quote}
\item 矩阵A与标量k的乘法:
\begin{quote}
\begin{math}
A*k=k*A=k*\left|
\begin{array}{cccc}
a_{11} & a_{12} & \cdots & a_{1m} \\
a_{21} & a_{22} & \cdots & a_{2m} \\
\vdots & \vdots & \ddots & \vdots \\
a_{n1} & a_{n2} & \cdots & a_{nm} \\
\end{array}
   \right| 
=
\left|
\begin{array}{cccc}
k*a_{11} & k*a_{12} & \cdots & k*a_{1m} \\
k*a_{21} & k*a_{22} & \cdots & k*a_{2m} \\
\vdots & \vdots & \ddots & \vdots \\
k*a_{n1} & k*a_{n2} & \cdots & k*a_{nm} \\
\end{array}
   \right|
\end{math}
\end{quote}
\item 矩阵的乘法,假设$m\times n$的矩阵A与矩阵$n\times v$的矩阵B相乘得$m\times v$的矩阵C,其中:
\begin{quote}
\begin{math}
c_{ij}=A_{i1}B_{1J}+A_{i2}B_{2j}+\cdots{}+A_{in}B_{nj}=\sum_{r=1}^{n}{A_{ir}B_{rj}}
\end{math}
\end{quote}
\item 矩阵的运算定律,假没$m\times n$矩阵A、B和C,以及标量k
\begin{enumerate}
\item $A+B=B+A$(加法交换律)
\item $A+(B+C)=(A+B)+C$(加法结合律)
\item $A*(B*C)=(A*B)*C$(乘法结合律)
\item $A*(B+C)=A*B+A*C$(分配律)
\item $k*(A+B)=k*A+k*B$(分配律)
\end{enumerate}
\end{enumerate}

\subsection{应用矩阵的变化}
\subsubsection{平移}
在3D空间中执行平移,需要将点P(x,y,z)平移到新的位置$P^{'}$(x+dx,y+dy,z+dz),用$4\times 4$执行钜阵变化为:
\begin{quote}
\begin{math}
M_{t}=\left|
\begin{array}{cccc}
1&0&0&0 \\
0&1&0&0 \\
0&0&1&0 \\
dx&dy&dz&1 \\
\end{array}
\right|
\end{math} 
\end{quote}
给定p=[x y z 1],执行如下平移变换:
\begin{quote} 
\begin{math}
p^{'}=p*M_{t}=[\begin{array}{cccc}x &y &z &1 \end{array}]* \left|
\begin{array}{cccc}
1&0&0&0 \\
0&1&0&0 \\
0&0&1&0 \\
dx&dy&dz&1 \\
\end{array}
\right|
\end{math} \\

\begin{math}
=[ \begin{array}{cccc} (x+1*dx) & (y+1*dx) & (z+1*dz) & 1  \end{array} ]
\end{math}  \\

\begin{math}
=[ \begin{array}{cccc} (x+dx) & (y+dx) & (z+dz) & 1  \end{array} ]
\end{math}  \\

\begin{math}
=p^{'}(x+dz,y+dy,z+dz)
\end{math}
\end{quote}
平移矩阵$M_{t}$的逆矩阵为$M_{t}^{-1}$:
\begin{quote}
\begin{math}
M_{t}^{-1}=\left|
\begin{array}{cccc}
1&0&0&0 \\
0&1&0&0 \\
0&0&1&0 \\
-dx&-dy&-dz&1 \\
\end{array}
\right|
\end{math}
\end{quote}
\subsubsection{缩放}
要相对于原点缩放某个点,只需要将点p(x,y,z)的各个分量分别乘以x轴,y轴,z轴的缩放因子sx,sy,sz。缩放矩阵为:
\begin{quote}
\begin{math}
M_{s}=\left|
\begin{array}{cccc}
sx & 0 & 0 & 0 \\
0 & sy & 0 & 0 \\
0 & 0 & sz & 0 \\
0 & 0 & 0 & 1  \\
\end{array}
\right|
\end{math}
\end{quote}
给定p=[$\begin{array}{cccc}x & y & z& 1\end{array}$],执行变化如下:
\begin{quote}
\begin{math}
p^{'}=p*M_{s}=[\begin{array}{cccc} x & y & z & 1 \end{array} ]* 
\left|
\begin{array}{cccc}
sx & 0 & 0 & 0 \\
0 & sy & 0 & 0 \\
0 & 0 & sz & 0 \\
0 & 0 & 0  & 1 \\
\end{array}
\right| \\
=[\begin{array}{cccc} x*sx & y* sy & z*sz & 1 \end{array} ]
\end{math}
\end{quote}
缩放矩阵$M_{s}$的逆矩阵$M_{s}^{-1}$为:
\begin{quote}
\begin{math}
M_{s}^{-1}=\left|
\begin{array}{cccc}
\frac{1}{sx} & 0 & 0 & 0 \\
0 & \frac{1}{sy} & 0 & 0 \\
0 & 0 & \frac{1}{sz} & 0 \\
0 & 0 & 0 & 1 \\
\end{array}
\right| 
\end{math}
\end{quote}
\subsubsection{旋转}
在3D坐标系中,可以绕三个轴进行旋转:x轴,y轴,z轴:
\begin{enumerate}
\item 绕x轴旋转,矩阵为:
\begin{quote}
\begin{math}
M_{x}=\left|
\begin{array}{cccc}
1 & 0 & 0 & 0 \\
0 & cos\theta & -sin\theta & 0 \\
0 & sin\theta & cos\theta & 0 \\
0 & 0 & 0 & 1 \\
\end{array}
\right|
\end{math}
\end{quote}
\item 绕y轴旋转,矩阵为:
\begin{quote}
\begin{math}
M_{y}=\left|
\begin{array}{cccc}
cos\theta & 0 & -sin\theta & 0  \\
0 & 1 & 0 & 0 \\
sin\theta & 0 & cos\theta & 0 \\
0 & 0 & 0 & 1 \\
\end{array}
\right| 
\end{math}
\end{quote}
\item 绕z轴旋转,矩阵为:
\begin{quote}
\begin{math}
M_{z}=\left|
\begin{array}{cccc}
cos\theta & -sin\theta & 0 & 0 \\
sin\theta & cos\theta & 0 & 0 \\
0  & 0 & 1 & 0 \\
0 & 0 & 0 & 1 \\
\end{array}
\right|
\end{math}
\end{quote}
\end{enumerate}
给定点p=[$\begin{array}{cccc} x & y & z & 1 \end{array} $ ],绕z轴旋转角度$\theta$:
\begin{quote}
\begin{math}
p^{'}=p*M_{z}=[\begin{array}{cccc} x & y & z & 1 \end{array} ]* 
\left|
\begin{array}{cccc}
cos\theta & -sin\theta & 0 & 0  \\
sin\theta & cos\theta & 0 & 0 \\
0 & 0 & 1 & 0 \\
0 & 0 & 0 & 1 \\
\end{array}
\right| \\
p^{'}=[\begin{array}{cccc} (x*cos\theta +y*sin\theta) & (x*sin\theta -y*cos\theta) & z & (1*1) \end{array} ] \\
p^{'}=( (x*cos\theta + y*sin\theta) , (x*sin\theta - y*cos\theta) , z ) 
\end{math}
\end{quote}

$M_{x}$,$M_{y}$和$M_{z}$的逆矩阵计算方法为:例如要计算$M_{z}$的逆矩阵,可以使用两种方法,一种方法是基于几何学,别一种方法基于线性代数。几何学的计算方法学为,例如将物体绕z轴旋转角度$\theta$,要将它恢复到原来的位置,只需将它旋转角度$-\theta$即可。因此要计算旋转矩阵的逆矩阵,只需将旋转矩阵中的$\theta$替换为$-\theta$,因此逆矩阵$M_{x}^{-1}$,$M_{y}^{-1}$和$M_{z}^{-1}$如下:
\begin{enumerate}
\item 矩阵$M_{x}$的逆矩阵$M_{x}^{-1}$为:
\begin{quote}
\begin{math}
M_{x}^{-1}=\left|
\begin{array}{cccc}
1 & 0 & 0 & 0 \\
0 & cos-\theta & -sin-\theta & 0 \\
0 & sin-\theta & cos-\theta & 0 \\
0 & 0 & 0 & 1 \\
\end{array}
\right|
=\left|
\begin{array}{cccc}
1 & 0 & 0 & 0 \\
0 & cos\theta & sin\theta & 0 \\
0 & -sin\theta & cos\theta & 0 \\
0 & 0 & 0 & 1 \\
\end{array}
\right|
\end{math}
\end{quote}
\item 矩阵$M_{y}$的逆矩阵$M_{y}^{-1}$为:
\begin{quote}
\begin{math}
M_{y}=\left|
\begin{array}{cccc}
cos-\theta & 0 & -sin-\theta & 0  \\
0 & 1 & 0 & 0 \\
sin-\theta & 0 & cos-\theta & 0 \\
0 & 0 & 0 & 1 \\
\end{array}
\right| 
=\left|
\begin{array}{cccc}
cos\theta & 0 & sin\theta & 0  \\
0 & 1 & 0 & 0 \\
-sin\theta & 0 & cos\theta & 0 \\
0 & 0 & 0 & 1 \\
\end{array}
\right| 
\end{math}
\end{quote}
\item 矩阵$M_{z}$的逆矩阵$M_{z}^{-1}$为:
\begin{quote}
\begin{math}
M_{z}^{-1}=\left|
\begin{array}{cccc}
cos-\theta & -sin-\theta & 0 & 0  \\
sin-\theta & cos-\theta & 0 & 0 \\
0 & 0 & 1 & 0 \\
0 & 0 & 0 & 1 \\
\end{array}
\right| 
=
\left| 
\begin{array}{cccc}
cos\theta & sin\theta & 0 & 0 \\
-sin\theta & cos\theta & 0 & 0 \\
0 & 0 & 1 & 0 \\
0 & 0 & 0 & 1 \\
\end{array}
\right|
\end{math}
\end{quote}
\end{enumerate}

\section{四元数}
\subsection{定义}
复数是由实数加上元素i组成,其中$i^{2}=-1$,相似地,四元数都是由实数加上三个元素i、j、k组成,而且它们有如下关系:
\begin{quote}
\begin{math}
i^{2}=j^{2}=k^{2}=ijk=-1 
\end{math}
\end{quote}
每个四元数都是1、i、j、k的线性组合,即是四元数一般可表示为:
\begin{quote}
$q=a+bi+cj+dk$
\end{quote}
另一种表示法采用标量与矢量相结合,例如:
\begin{quote}
\begin{math}
q=a+\vec{u}=a+bi+cj+dk \\
p=t+\vec{v}=t+zi+yj+zk
\end{math}
\end{quote}
其中$\vec{u}$表示矢量$<b,c,d>$,而$\vec{v}$表示矢量$<x,y,z>$

\subsection{性质}
\subsubsection{共轭四元数}
对于四元数q,用$q^{*}$表示其共轭四元数,其计算方法只需要将虚数分量的符号反过来即可,给定四元数q:
\begin{quote}
$ q=a+\vec{u}=a+bi+cj+dk$
\end{quote}
其共轭四元数$q^{*}$为:
\begin{quote}
$ q^{*}=a-\vec{u}=a-bi-cj-dk$
\end{quote}
\subsubsection{四元数的范数}
范数表示四元数的长度信息,给定四元数h:
\begin{quote}
$ h=a+\vec{u}=a+bi+cj+dk$
\end{quote}
其范数为:
\begin{quote}
$|h|=\sqrt{h\cdot h^{*}}=\sqrt{a^{2}+b^{2}+c^{2}+d^{2}}$
\end{quote}


\subsection{四元数运算}
假设有两个四元数:
\begin{quote}
\begin{math}
q=a+\vec{u}=a+bi+cj+dk \\
p=t+\vec{v}=t+zi+yj+zk
\end{math}
\end{quote}
\subsubsection{加法}
两个四元数相加,跟复数,矢量和矩阵一样,只需要将不同的元数加起来即可:
\begin{quote}
\begin{math}
p+q=a+t+\vec{u}+\vec{v}=(a+t)+(b+x)i+(c+y)j+(d+z)k
\end{math}
\end{quote}
\subsubsection{加法逆元素与加法恒等元}
四元数q的加逆元数是这样一个数:将它与q相加时,结果为零,它就是-q,为:
\begin{quote}
$-q=-a-\vec{u}=-a-bi-ci-dk $
\end{quote}
加法恒等元(或四元数数学中的\verb|"0"|)为:
\begin{quote}
$z=0+<0,0,0>=0+0*i+0*j+0*k $
\end{quote}
\subsubsection{乘法}
两个四元数之间的非可换乘积通常被称为格拉斯曼积,元系之间的乘积可以跟随以下的乘法表, 四元数的单位元的乘法构成了八阶四元群,$Q_{8}$
\begin{quote}
\begin{tabular}{c|c|c|c|c}
 & 1 & i & j & k \\ 
\hline 
1 & 1 & i & j & k \\
\hline 
i & i & -1 & k & -j \\
\hline 
j & j & -k & -1 & i \\
\hline 
k & k & j & -i & -1 \\
\end{tabular}
\end{quote}
乘法如下:
\begin{quote}
\begin{math}
pq=at-\vec{u}\cdot\vec{v}+a\vec{v}+t\vec{u}+\vec{v}\times\vec{u} \\
pq=(at-bx-cy-dz)+(bt+ax+cz-dy)i+(ct+ay+dx-bz)j+(dt+za+by-xc)k
\end{math}
\end{quote}
\subsubsection{点积}
点积也叫做欧几里得内积,四元数的点积等同于一个四维矢量的点积,点积的值是p中每个元数的数值与q中相应元数的数值的乘积的和,结果为一个标量。
\begin{quote}
$p\cdot q=at+\vec{u}\cdot{v}=at+bx+xy+dz $
\end{quote}
点积也可以用格拉斯曼积的形式表示:
\begin{quote}
\begin{math}
p\cdot q = \frac{p^{*}q+q^{*}p}{2}
\end{math}
\end{quote}

\subsubsection{倒数}
如果四元数p与另一个四元数q,它们之间满足下列等式:
\begin{quote}
$qp=1=pq$
\end{quote}
则称q为p的倒数,对于四元数q,其倒数记为$q^{-1}$,其值为:
\begin{quote}
$q^{-1}=\frac{h^{*}}{|h|^{2}}$
\end{quote}



























\end{CJK}
\end{document}

