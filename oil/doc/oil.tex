\documentclass[a4paper]{article}
\usepackage{CJK}
\usepackage{indentfirst}
\usepackage{graphicx}
\usepackage{bibentry,natbib}
\usepackage{fancyhdr}
\usepackage{lastpage}

%\usepackage{iffalse} 
%\usepackage[top=2.5cm, bottom=2.5cm, left=2.5cm, right=2.5cm]{geometry}

\begin{document}
\begin{CJK}{UTF8}{song}
\title{Object Orient Information Language}
\author{Nosiclin(nosiclin@foxmail.com)}
\date{2012-07-23}
\maketitle

\section{简介}
Object Orient Information Language 简称 OIL,类似于XML以结构化的形式来组织信息,不同的是在OIL中加入两个高级的概念:数组与字典。数组是一种顺序性容器,字典为一种关联容器,每个元素由键和值组成。在字典中可以根据提贡的键获得其相应的值。从面向对象的方面角度来看,可以把字典看作为对象,字典的键可以看做对象的属性,键所关联的值可以看作是对象属性的值。这是取名Object Orient Information Language的原因。

在OIL中最基本的数据类型为字符串,它与数组,字典统称为OIL对象,下面依次介绍它们三个。
\section{字符串}
字符串是组成OIL的最基本的元素,也是唯一一个基本元素,在OIL中没有浮点数,整数,长整数的概念。这样做的原因是保持OIL设计的简洁性,整数,浮点数等其它信息都用字符串表示,具体的数据类型交给上层应用来识别与处理。上层应用可以调用OIL中提贡的基本函数来获得目标类型,例如把字符串转换为整型函数ParseInt,转换为浮点为ParseFloat,或者根据上层应用使用自己的类型转换函数,例如在OIL存储了为192.168.1.1的ip地址,ip地址以字符串表示,但OIL中没有提贡转换为IP的函数,应用可以根据自己的需要编写函数ParseIp。

\subsection{字符串的格式}
在OIL中,字符串有三种表示法:
\begin{enumerate}
 \item 直接表示法:由多个字符组成,但要由字符之间没有空白符(制表符和空格),同时也要求字符为非特殊符号,字符前的和字符后的空白符都被忽略。例如:160kg 或者 32.4cm 。
\item 双引号:字符串以双引号开头和双引号结尾,中间有零个或多个字符组成,一些在键盘不能直接输入的字符,可以通过转义符\verb|'\'|转义。例如:
\begin{quote}
\verb|"My Name Is Tom \n I Have A Good Dog"|
\end{quote}
 或者是 
\begin{quote}
\verb|"This \t\t BOOk Is Not Good For You"|
\end{quote}
其中字符 \verb|\n| 和字符 \verb|\t| 会被转义为换行符和制表符。
\item 单引号:字符串以单引号开头和单引号结尾,字符串中的所有字符都保留原意。例如:
\begin{quote}
\verb|'Hello \t ,nice to meet you'|
\end{quote}
其中字符 \verb|\t| 保留原意,不会被转义。
\end{enumerate}

\section{字典}
在OIL中,字典用花括号表示,在括号中,可以包含零个或多个字典元素,字典元素由键和值组成,表示为 键:值。字典元素之间使用换行符或者是逗号隔开,键的命名规则为以字母或下划线开头,后面可以接零个或多个字母、数字或者是下划线。当要描述一个男孩,它的名字为Tom,身高为160cm,体重为50kg,年龄为23岁用OIL表示为:
\begin{quote}
\begin{verbatim}
{
    name:Tom
    height:160cm
    weight:50kg
    age:23
}
\end{verbatim}
\end{quote}
在字典中,元素之间的顺序对字典没有影响,元素之间的顺序可以随意交换,它们的表示是等价的。字典元素的值可以为字符串,数组,字典,当为字符串时,要求字符串的开始字符比须与键在同一行。例如下面的为错误表示:
\begin{quote}
\begin{verbatim}
{
    name:       <-----错误表示法,要求它们在name与Tom在同一行
         Tom
}
\end{verbatim}
\end{quote}
当元素的值为数组的字典时,为了书写方面与视觉美感,不要求它们在同一行,例如下面的表示为合法的:
\begin{quote}
\begin{verbatim}
{
    address:"Meet Room A"
    people:     <----不在同一行,但是合法表示
    {          
       name:Tom
       weight:50kg
    }
}
\end{verbatim}
\end{quote}
字典元素之间也可以使用逗号分隔开来,例如:
\begin{quote}
\begin{verbatim}
{ name:Tom,height:160cm,weight:50kg,age:23}
\end{verbatim}
\end{quote}



\section{数组}
另一个容器为数组,数组用方括号表示,括号中可以包含零个或多个数组元素。例如包含10个数字的数组:
\begin{quote}
\begin{verbatim}
[1,3,4,5,2,6,9.8.7.0]
\end{verbatim}
\end{quote}
数组中,元素之间的顺序得到了保留,数组的元素可以为字符串,数组,字典。并且元素之间可以通过换行符隔开,例如:
\begin{quote}
\begin{verbatim}
[
  First
  Second
  Third
]
\end{verbatim}
\end{quote}

\section{变量}
很多时候,同一个数值会被多次使用,但是一开始不能准确的给出,需要经过多次测试后才能确定,但是每次更改都需要改动多个地方。为了解决这个问题,同时也是为了书写方面,OIL支持变量定义,但变量需要在OIL文件的开头全部定义,不能出现在OIL文件中间,在变量定义后,可以在其它地方引用该变量。
\subsection{变量的格式}
变量的定义由两部分组成,变量名和变量值,格式为:
\begin{quote}
\begin{verbatim}
&变量名 : 变量值
\end{verbatim}
\end{quote}
变量的定义规则与字典的元素类似,只不过在前面加上了符号 \verb|&| 。在其它地方可以使用:
\begin{quote}
\begin{verbatim}
 $变量名 
\end{verbatim}
\end{quote}
引用变量。例如:
\begin{quote}
\begin{verbatim}
&num=13234455
people:
{
    name:Tom
    ID:$num       <----引用变量num
}
\end{verbatim}
\end{quote}

\section{特殊字符}
在OIL中,以下字符为特殊字符,使用时需要注意:
\begin{quote}
\begin{verbatim}
{  } [  ]  ,  "  '  :  &  $ #
\end{verbatim}
\end{quote}

\section{OIL文法}
This part description the grammar of the OIL.
\begin{quote}
\begin{verbatim}
%start oil
oil=dict_body
dict_body={dict_stmt} [dict_entry]
dict_stmt=dict_entry delimiter | null_stmt
dict_entry=name ":" object | name ":" "\n" {"\n"} container
name=$IDENTIFIER | IDENTIFIER 
container=array | dict 
object=STRING | array |dict 
array="[" array_body "]"
array_body={ array_stmt } [ array_stmt ] 
array_stmt=array_entry delimiter | null_stmt
array_entry=object 
dict="{" dict_body "}"
delimiter="\n" | ","

\end{verbatim}
\end{quote}

\end{CJK}
\end{document}